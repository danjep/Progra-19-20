\documentclass[12pt]{article}
\usepackage[utf8]{inputenc}
\usepackage{geometry}
\usepackage{svg}
\usepackage{float}
\usepackage{caption}
\usepackage{amsmath,amsthm,amsfonts,amssymb,amscd}
\usepackage{fancyhdr}
\usepackage{titlesec}
\usepackage{xparse}
\usepackage{listings}
\usepackage{tikz}
\usepackage{mathtools}
\usepackage[ngerman]{babel}
\usetikzlibrary{shapes.geometric, arrows}
\pagestyle{empty}
\titleformat*{\section}{\large\bfseries}

%
\geometry{
 a4paper,
 total={170mm,240mm},
 left=20mm,
 top=30mm,
 }

\date{}
%Bitte ausfüllen
\newcommand\course{Programmierung, Gruppe 16}
\newcommand\hwnumber{6}
\newcommand\Name{Maximilian Petri, 405602}
\newcommand\Neptun{Danje Petersen, 379748}

%Matheinheiten
\newcommand\m{\:\textrm{m}}
\newcommand\M{\:\Big[\textrm{m}\Big]}
\newcommand\mm{\:\textrm{mm}}
\newcommand\MM{\:\Big[\textrm{mm}\Big]}
\newcommand\un{\underline}
\newcommand\s{\:\textrm{s}}
\newcommand\bS{\:\Big[\textrm{S}\Big]}
\newcommand\ms{\:\frac{\textrm{m}}{\textrm{s}}}
\newcommand\MS{\:\Big[\frac{\textrm{m}}{\textrm{s}}\Big]}
\newcommand\mss{\:\frac{\textrm{m}}{\textrm{s}^2}}
\newcommand\MSS{\:\Big[\frac{\textrm{m}}{\textrm{s}^2}\Big]}
\DeclarePairedDelimiter\ceil{\lceil}{\rceil}
\DeclarePairedDelimiter\floor{\lfloor}{\rfloor}

%Einstellungen für lstlisting
\lstset{
basicstyle=\fontsize{9}{10}\selectfont\ttfamily, 
keywordstyle=\color{purple}\bfseries,           % keywords purple
commentstyle=\color{gray},                     % white comments
stringstyle=\color{blue},                       % strings blue
showstringspaces=false,
language= java}                          % typewriter type for strings                       


%Bitte nicht einstellen

\renewcommand{\thesection}{Aufgabe }
\renewcommand{\thesubsection}{\alph{subsection})}
\renewcommand{\thesubsubsection}{\hspace{0.8cm}\roman{subsubsection})}
\renewcommand{\figurename}{Abbildung}
\renewcommand{\tablename}{Tabelle}
\pagestyle{fancyplain}
\headheight 35pt
\lhead{\Name\\\Neptun}
\chead{\textbf{\Large Hausaufgabe \hwnumber}}
\rhead{\course \\ \today}
\lfoot{}
\cfoot{}
\rfoot{\small\thepage}
\headsep 1.5em

\begin{document}

\section{2)}
\begin{lstlisting}
    /**
     * Diese Methode zeichnet einen Pythagors Baum auf ein Canvas mit verschiedenen
     * Einstellungsmoeglichkeiten.
     * 
     * @param c            Das Canvas auf dem der Pythagoras Baum gezeichnet werden
     *                     soll.
     * @param level        Die Tiefe der Rekursion.
     * @param length       Die Groesse des ersten Rechtecks.
     * @param leftAngle    Die Groesse des linken Winkels.
     * @param rightAngle   Die Groesse des rechten Winkels.
     * @param switchLength Die Laenge ab welcher alle groesseren Rechtecke Braun 
     *                     seien sollen.
     */
    private static void paintPythagorasTree(
            Canvas c,
            int level,
            double length,
            int leftAngle,
            int rightAngle,
            int switchLength) {

        // check for level
        if (level < 1) {
            return;
        }

        // set color
        if (length <= switchLength) {
            c.chooseColor(c.GREEN);
        } else {
            c.chooseColor(c.BROWN);
        }

        // draw Square
        c.square(length);

        // Right
        double rightLength = calculateLength(leftAngle, rightAngle, length);
        c.move(length / 2, -length / 2);
        c.rotate(rightAngle);
        c.move(-rightLength / 2, -rightLength / 2);

        paintPythagorasTree(c, level - 1, rightLength, leftAngle, rightAngle, switchLength);

        c.move(rightLength / 2, rightLength / 2);
        c.rotate(-rightAngle);
        c.move(-length / 2, length / 2);

        // Left
        double leftLength = calculateLength(rightAngle, leftAngle, length);
        c.move(-length / 2, -length / 2);
        c.rotate(-leftAngle);
        c.move(leftLength / 2, -leftLength / 2);

        paintPythagorasTree(c, level - 1, leftLength, leftAngle, rightAngle, switchLength);

        c.move(-leftLength / 2, leftLength / 2);
        c.rotate(leftAngle);
        c.move(length / 2, length / 2);
    }

    private static double calculateLength(int alpha, int beta, double length) {
        return (Math.sin(Math.toRadians(alpha)) * length) / 
                (Math.sin(Math.toRadians(180 - alpha - beta)));
    }
\end{lstlisting}

\section{4)}
\begin{center}
    VPL Abgabe von Maximilian Petri
\end{center}

\end{document}