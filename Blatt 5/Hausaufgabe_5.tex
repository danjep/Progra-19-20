\documentclass[12pt]{article}
\usepackage[utf8]{inputenc}
\usepackage{geometry}
\usepackage{svg}
\usepackage{float}
\usepackage{caption}
\usepackage{amsmath,amsthm,amsfonts,amssymb,amscd}
\usepackage{fancyhdr}
\usepackage{titlesec}
\usepackage{xparse}
\usepackage{listings}
\usepackage{tikz}
\usepackage{mathtools}
\usepackage[ngerman]{babel}
\usetikzlibrary{shapes.geometric, arrows}
\pagestyle{empty}
\titleformat*{\section}{\large\bfseries}

%
\geometry{
 a4paper,
 total={170mm,240mm},
 left=20mm,
 top=30mm,
 }

\date{}
%Bitte ausfüllen
\newcommand\course{Programmierung, Gruppe 16}
\newcommand\hwnumber{5}
\newcommand\Name{Maximilian Petri, 405602}
\newcommand\Neptun{Danje Petersen, 379748}

%Matheinheiten
\newcommand\m{\:\textrm{m}}
\newcommand\M{\:\Big[\textrm{m}\Big]}
\newcommand\mm{\:\textrm{mm}}
\newcommand\MM{\:\Big[\textrm{mm}\Big]}
\newcommand\un{\underline}
\newcommand\s{\:\textrm{s}}
\newcommand\bS{\:\Big[\textrm{S}\Big]}
\newcommand\ms{\:\frac{\textrm{m}}{\textrm{s}}}
\newcommand\MS{\:\Big[\frac{\textrm{m}}{\textrm{s}}\Big]}
\newcommand\mss{\:\frac{\textrm{m}}{\textrm{s}^2}}
\newcommand\MSS{\:\Big[\frac{\textrm{m}}{\textrm{s}^2}\Big]}
\DeclarePairedDelimiter\ceil{\lceil}{\rceil}
\DeclarePairedDelimiter\floor{\lfloor}{\rfloor}

%Einstellungen für lstlisting
\lstset{
basicstyle=\fontsize{9}{10}\selectfont\ttfamily, 
keywordstyle=\color{purple}\bfseries,           % keywords purple
commentstyle=\color{gray},                     % white comments
stringstyle=\color{blue},                       % strings blue
showstringspaces=false,
language= java}                          % typewriter type for strings                       


%Bitte nicht einstellen

\renewcommand{\thesection}{Aufgabe }
\renewcommand{\thesubsection}{\alph{subsection})}
\renewcommand{\thesubsubsection}{\hspace{0.8cm}\roman{subsubsection})}
\renewcommand{\figurename}{Abbildung}
\renewcommand{\tablename}{Tabelle}
\pagestyle{fancyplain}
\headheight 35pt
\lhead{\Name\\\Neptun}
\chead{\textbf{\Large Hausaufgabe \hwnumber}}
\rhead{\course \\ \today}
\lfoot{}
\cfoot{}
\rfoot{\small\thepage}
\headsep 1.5em

\begin{document}

\section{2)}
\begin{center}
    VPL Abgabe von Maximilian Petri
\end{center}

\section{4)}
\begin{center}

\subsection{}
\textbf{4.0}\\
Es wird der erste Konstruktor benutzt, und somit wird die 4 zum Typ Double umgewandelt und b1.d zugewiesen.

\subsection{}
\textbf{13}\\
7d und 8L, also Zahlen vom Typ double und Long werden übergeben. Es gibt keine Funktion f mit genau diesen Datentypen. Da es nach unboxing des Long und autoboxing des double den Typen aus f(Double x, long y) entspricht wird diese Funktion ausgeführt.

\subsection{}
\textbf{11}\\
Da der Funktion ein double und ein int übergeben werden wird die f(double x, int y) aufgerufen und deshalb 11 zurückgegeben.

\subsection{}
\textbf{12}\\
Es werden ein int und ein Long übergeben. Es existiert keine Funktion f mit exakt diesen Übergabetypen. Durch unboxing und implizites casting wird aus dem Long ein float, wodurch f(int x, float y) benutzt werden kann. 

\subsection{}
\textbf{6.0}\\
Für b2 wird der erste Konstruktor benutzt, da man durch unboxing von b1.i1 zu 4 Werten des Typ int kommt, beim zweiten Konstruktor bräuchte man implizites casting von int zu double und int zu float und außerdem autoboxing von double zu Double.
Da b2.f als float definiert ist, wird die übergebene 6 somit zur 6.0.

\subsection{}
\textbf{11}\\
Mit b1.f und b1.i2 werden ein float und ein int übergeben. Durch einmaliges implizites casting wird aus dem float ein double und es wird f(double x, int y) benutzt.

\subsection{}
\textbf{1.5}\\
Für b3 muss der zweite Konstruktor benutzt werden, da aus der 1.5 nicht implizit ein int gecastet werden kann.
Somit wird 1.5 b3.d zugewiesen.

\subsection{}
\textbf{8.0}\\
Da implizites casting priorität gegenüber autoboxing hat, wird aus dem Integer b1.i1 eher ein double als ein Float.

\subsection{}
\textbf{7.0}\\
Da mit Float.valueOf(18) ein Wert des Typ Float übergeben wird, wird die entsprechende Funktion mit dem Übergabeparameter Float x benutzt.

\subsection{}
\textbf{11}\\
Bei b2.g(19f) wird ein Wert vom Typ float zurückgegeben. Dieser kann dann leicht zu double gescastet werden und somit werden double und int übergeben.
\end{center}
\end{document}