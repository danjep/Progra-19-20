\documentclass[12pt]{article}
\usepackage[utf8]{inputenc}
\usepackage{geometry}
\usepackage{svg}
\usepackage{float}
\usepackage{caption}
\usepackage{amsmath,amsthm,amsfonts,amssymb,amscd}
\usepackage{fancyhdr}
\usepackage{titlesec}
\usepackage{xparse}
\usepackage{listings}
\usepackage{tikz}
\usepackage{mathtools}
\usepackage[ngerman]{babel}
\usetikzlibrary{shapes.geometric, arrows}
\pagestyle{empty}
\titleformat*{\section}{\large\bfseries}

%
\geometry{
 a4paper,
 total={170mm,240mm},
 left=20mm,
 top=30mm,
 }

\date{}
%Bitte ausfüllen
\newcommand\course{Programmierung, Gruppe 16}
\newcommand\hwnumber{3}
\newcommand\Name{Maximilian Petri, 405602}
\newcommand\Neptun{Danje Petersen, 379748}

%Matheinheiten
\newcommand\m{\:\textrm{m}}
\newcommand\M{\:\Big[\textrm{m}\Big]}
\newcommand\mm{\:\textrm{mm}}
\newcommand\MM{\:\Big[\textrm{mm}\Big]}
\newcommand\un{\underline}
\newcommand\s{\:\textrm{s}}
\newcommand\bS{\:\Big[\textrm{S}\Big]}
\newcommand\ms{\:\frac{\textrm{m}}{\textrm{s}}}
\newcommand\MS{\:\Big[\frac{\textrm{m}}{\textrm{s}}\Big]}
\newcommand\mss{\:\frac{\textrm{m}}{\textrm{s}^2}}
\newcommand\MSS{\:\Big[\frac{\textrm{m}}{\textrm{s}^2}\Big]}
\DeclarePairedDelimiter\ceil{\lceil}{\rceil}
\DeclarePairedDelimiter\floor{\lfloor}{\rfloor}

%Einstellungen für lstlisting
\lstset{
basicstyle=\fontsize{13}{14}\selectfont\ttfamily, 
keywordstyle=\color{purple}\bfseries,           % keywords purple
commentstyle=\color{gray},                     % white comments
stringstyle=\color{blue},                       % strings blue
showstringspaces=false,
language= java}                          % typewriter type for strings                       


%Bitte nicht einstellen

\renewcommand{\thesection}{Aufgabe }
\renewcommand{\thesubsection}{\alph{subsection})}
\renewcommand{\thesubsubsection}{\hspace{0.8cm}\roman{subsubsection})}
\renewcommand{\figurename}{Abbildung}
\renewcommand{\tablename}{Tabelle}
\pagestyle{fancyplain}
\headheight 35pt
\lhead{\Name\\\Neptun}
\chead{\textbf{\Large Hausaufgabe \hwnumber}}
\rhead{\course \\ \today}
\lfoot{}
\cfoot{}
\rfoot{\small\thepage}
\headsep 1.5em

\begin{document}

\section{2)}
\begin{lstlisting}
public class Selection {

    public static void selection(int[] a) {

        for (int n = 0; n < a.length; n++) {
            int i_min = n;

            // find i_min
            for (int i = n; i < a.length; i++) {
                if (a[i] < a[i_min]) {
                    i_min = i;
                }
            }

            // switch i_min with n-1
            int temp = a[n];
            a[n] = a[i_min];
            a[i_min] = temp;
        }
    }
}
\end{lstlisting}

\pagebreak
%%%%%%%%%%%%%%%%%%%%%%%%%%%%%%%%%%%%%%%%%%%%%%%%%%%%%%%%%%%%%%%%%%%%%%%

\section{4)}
\subsection{}
\begin{center}
    Partielle Korrektheit:\\
    Die Schleifeninvariante ist $ \langle res = x \epsilon \{a[j] \ | \ 0 \leq j \leq i\} \rangle$ .
    \bigbreak
    \begin{tabular}{l}
        $\langle true \rangle$\\
        $\langle 0 = 0 \wedge false = false \rangle$\\
        \quad \textbf{i = 0;}\\
        $\langle i = 0 \wedge false = false \rangle$\\
        \quad \textbf{res = false;}\\
        $\langle i = 0 \wedge res = false \rangle$\\
        $\langle res = x \epsilon \{a[j] \ | \ 0 \leq j \leq i\} \land i \leq a.length \rangle$\\
        \quad \textbf{while (i \pmb{$<$} a.length) \{}\\
        \quad \quad $\langle res = x \epsilon \{a[j] \ | \ 0 \leq j \leq i\} \wedge i \leq a.length \land i < a.length \rangle$\\
        \quad \quad \quad \textbf{if (x == a[i]) \{}\\
        \quad \quad \quad \quad $\langle res = x \epsilon \{a[j] \ | \ 0 \leq j \leq i\} \wedge i \leq a.length \wedge x = a[i] \rangle$\\
        \quad \quad \quad \quad $\langle true = x \epsilon \{a[j] \ | \ 0 \leq j \leq i+1\} \wedge i+1 \leq a.length \rangle$\\
        \quad \quad \quad \quad \quad \textbf{res = true;}\\
        \quad \quad \quad \quad $\langle res = x \epsilon \{a[j] \ | \ 0 \leq j \leq i+1\} \wedge i+1 \leq a.length \rangle$\\
        \quad \quad \quad \textbf{\}}\\
        \quad \quad $\langle res = x \epsilon \{a[j] \ | \ 0 \leq j \leq i+1\} \wedge i+1 \leq a.length \rangle$\\
        \quad \quad \quad \textbf{i = i + 1;}\\
        \quad \quad $\langle res = x \epsilon \{a[j] \ | \ 0 \leq j \leq i\} \wedge i \leq a.length \rangle$\\
        \quad \textbf{\}}\\
        $\langle res = x \epsilon \{a[j] \ | \ 0 \leq j \leq i\} \wedge i \leq a.length \wedge \neg(i < a.length) \rangle$\\
        $\langle res = x \epsilon \{a[j] \ | \ 0 \leq j \leq a.length-1\} \rangle$\\
    \end{tabular}
    \bigbreak
    Damit ist unter der Vorbedingung die partielle Korrektheit des Programms gezeigt.
\end{center}

\subsection{}
\begin{center}
    Terminierung:\\
    Die Variante der Schleife ist $a.length - i$ .
    \bigbreak
    \begin{tabular}{l}
        $\langle a.length - i = m \land i < a.length \rangle$\\
        $\langle a.length - (i+1) < m \land i < a.length \rangle$\\
        \quad \textbf{if (x == a[i]) \{}\\
        \quad \quad$\langle a.length - (i+1) < m \land i < a.length \land x = a[i] \rangle$\\
        \quad \quad$\langle a.length - (i+1) < m \rangle$\\
        \quad \quad \quad \textbf{res = true;}\\
        \quad \quad$\langle a.length - (i+1) < m \rangle$\\
        \quad \textbf{\}}\\
        $\langle a.length - (i+1) < m \rangle$\\
        \quad \textbf{i = i + 1;}\\
        $\langle a.length - i < m \rangle$\\
    \end{tabular}
    \bigbreak
    Damit ist gezeigt, dass das Programm unter der Vorbedingung terminiert.
\end{center}

\end{document}